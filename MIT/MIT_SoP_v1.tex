\documentclass[a4paper,10pt]{report}
\hyphenation{mathe-matics under-graduate know-ledge}
\usepackage{fullpage}
\usepackage{fancyhdr}
\pagestyle{fancy}
\fancyhead[L]{\small Kittipat Apicharttrisorn}
\fancyhead[C]{\bfseries  \large Statement of Purpose}
\fancyhead[R]{\small \degree \space (EECS) \\ \university}
\fancyfoot[C]{\thepage}
\topmargin = 10pt
\headheight = 14pt
\headsep = 14pt
\footskip = 30pt
%\usepackage{helvetica}
\newcommand{\university}{Massachusetts Institute of Technology}
\newcommand{\department}{Electrical Engineering and Computer Science Department}
\newcommand{\uniabbre}{MIT}
\newcommand{\degree}{Doctor of Philosophy}
\newcommand{\labfirst}{N/A}
\newcommand{\labfirstabbre}{N/A}
\begin{document}
%Please submit your statement of purpose. This one- or two-page statement states your reasons for undertaking graduate work and explains your academic interests, including their relation to your undergraduate study and professional goals. Include your full name and your proposed field of study at the top of each page.

\vspace{0.4cm}
This statement of purpose is intended for use with my application to the \degree \space graduate program at the \department, \university. This document starts by portraying my educational background, of both the Bachelor and Master's degrees. Then, it briefly states my professional experience during the employment as a systems engineer and explains my research experience during the Master's degree study. Then my interest in \uniabbre's teaching and research is elaborated and finally my future plans after Ph.D. graduation are described. After finishing reading this statement of purpose, the committee will learn why I am qualified to be an excellent student of the program, what motivates me to pursue the doctoral degree at \uniabbre \space and why it is so important for my future profession that I earn this degree.

\vspace{0.2cm}
During my undergraduate study, in addition to a number of Electrical Engineering subjects, I studied a wide range of mathematical subjects including four Calculus courses, a course on Probability, and another on Linear Algebra and Complex Numbers, all of which are basic principles of computer science. Moreover, I passed two courses on computer programming, data structures and algorithms, which are the crucial knowledge of a successful computer scientist. For the Master's degree study, I passed eight credited computer science graduate courses. I received four As from the following courses - Information Systems Architecture, Artificial Intelligence, Distributed Systems, and Advanced Topics on Computer Networks (Multimedia, Wireless and Adhoc Networks). The other four B+ subjects were Computer Algorithms, Theory of Computation, Embedded Systems, and Database Management Systems. Moreover, I studied two non-credit courses - namely Computer Security and Special Topics on Distributed Systems (Service Computing). In sum, I earn a solid foundation in computer science as a result of my undergraduate and graduate study. 

\vspace{0.2cm}
I also have seven-year professional experience working at Aeronautical Radio of Thailand or Aerothai, a state enterprise under the Ministry of Transport, Thailand. One of Aerothai's principal missions is to provide air navigation services or air traffic control within Thailand's airspace. Specifically, the department of air traffic data systems engineering is responsible for the provision and administration of data systems that support air traffic controllers' operations efficiently and continuously. At the department, my colleagues and I design, configure, and implement those systems by taking advantage of enterprise-graded information system products, mostly of the USA, such as HP and Dell servers, Oracle and Microsoft databases, Cisco network equipment, and VMWare's virtualization technology, etc. Therefore, I have witnessed how these innovative products help transform air traffic data systems into those with enhanced reliability and efficiency. This hand-on experience allows me to learn practical aspects of enterprise information systems with the safety-critical applications, and motivates me to study more deeply and broadly in computer science, a core foundation of computer-related products and services. Moreover, the exposure to these technologies encourages me to plan to further my study in the US.

\vspace{0.2cm}
Up to now, I have published three academic publications, two of which are in international conferences' proceedings and the other is in an ACM journal.  First, ``Energy-Efficient Gradient Time Synchronization for Wireless Sensor Networks'' was accepted for publication in the proceedings of the Second International Conference on Computational Intelligence, Communication Systems and Networks or CICSyN. In the paper, we designed an extended version of gradient time synchronization protocols that was more time-accurate and energy-efficient, while maintaining a ``gradient'' property. With the gradient property, geographically adjacent nodes are able to maintain minimal synchronization errors. Second, ``Desynchronization with an artificial force field for wireless networks'' was accepted to publish in ACM SIGCOMM's \textit{Computer Communication Review}. The desynchronization problem was analogous to a resource allocation problem in which nodes cooperated to take turns accessing to the same resource. In this paper, we provided a prove of convexity of this problem. Additionally, we designed a desynchronization protocol, inspired by electromagnetic force field, that performed in a distributed manner, better scaled with network sizes and densities and produced less desynchronization errors. The first two papers were my work under the supervision of Assistant Professor Dr. Chalermek Intanagonwiwat. Third, in 2013, I had a change to work on a research project with Associate Professor Dr. Teerasit Kasetkasem of Kasetsart University. In this project, we used a signal processing technique to track a moving object in a field given binary sensor observations. In this paper, I was fully responsible for the manuscript preparation and partly for experimental simulation. Finally, the paper titled ``A Moving Object Tracking Algorithm Using Support Vector Machines in Binary Sensor Networks'', was finally accepted for publication in the proceedings of The 13th International Symposium on Communications and Information Technologies.

\vspace{0.2cm}
I desire to advance my study to a PhD in the US because of the following three main reasons. First and most importantly, I want to be a professional researcher in computer science in the future, either in an academic institution or in a research laboratory and a doctoral degree is an important precursor to the research profession. Second, I agree with Matt Welsh, previously a professor of Computer Science at Harvard University, about a PhD study. He suggests that ``You get an intense exposure to every subfield of Computer Science, and have to become the leading world's expert in the area of your dissertation work.'' For example, during my PhD study, I will have an opportunity to get exposed to a variety of academic subjects and research projects in computer science, such as Artificial Intelligence, Computer Graphics, Robotics, Databases, Systems, Software Engineering, and Computational Science, etc., all of which will considerably expand my intellectual horizons in computer science. Moreover, the PhD study will train me to be an expert in the field of my dissertation through the educational systems and processes, together with my assiduous and persevering efforts. Third, I am conscious that studying at a PhD level requires an academically vibrant environment which includes surroundings with brilliant students and faculty members, as well as accessible academic conferences and seminars. In my opinion, all of these are prevalent in the US educational systems and universities.

% the following paragraphs are specific to a particular program in a university. need to change for each submission to each program
\vspace{0.2cm}
I aspire to become a PhD student at \department, \university, a prestigious university in the US, because I am particularly interested in its teaching and research. To begin with the teaching, Assistant Professor Nate Foster's Network Programming is of my particular interest because network programming is a skill that I need to master in order to be a competent computer network researcher in the future. Network programming has a distinctive style because it entails several modules and components from application layers down to MAC and physical layers. Moreover, it is difficult to debug network programs because of distributed nodes inter-connected by error-prone and latency-inducing communication links. In addition, the "Seminar in Programming Languages" course will provide me with hands-on programming skills that are essential for my researcher career. Furthermore, Associate Professor Emin Gün Sirer's Operating Systems is also an interesting subject that educates students with theoretical and practical aspects of operating systems. Operating systems are indispensable middleware of today's computing systems that reliably and systematically links the application software with the computer hardware. Computer network research often involves modifying or testing low-level abstractions of operating systems; therefore, principles and practice of operating systems will give me a deeper understanding and stronger expertise in computer network research. In addition, ``Cornell Systems Lunch'' is a seminar that introduces students with novel and recent systems research. I look forward to attending and participating in this seminar because it will teach me to think critically and scientifically during the presentations and discussions. In sum, these four courses taught and driven by the faculty at the \department \space are just a few examples showing my interest in the teaching at \uniabbre .

\vspace{0.2cm}
The following are \uniabbre's faculty whose research projects interest and excite me. First, Associate Professor Emin Gün Sirer's publication ``On the Feasibility of Completely Wireless Datacenters'' proposes a novel networking model of data centers. Instead of attaching to wired networks, in this paper, completely wireless data center networking is investigated and the experimental results surprisingly suggest that it outperforms traditional wired networking in terms of bandwidth, latency, and fault tolerance. During the professional experience with aeronautical data systems, wired networks reduce the maintainability of such systems. For example, when we alter just a few network policies, we need to re-configure and re-wire a non-trivial cluster of systems. In completely wireless data centers, re-configuration alone would suffice to reflect these new policies. However, more work needs to be further investigated. For instance, how well these wireless networks work in a real data center running particular applications. 

\vspace{0.2cm}
Second, Assistant Professor Nate Foster is focusing his research on software-defined networks or SDNs. In my opinion, SDNs are the future of computer networks because it allows network administrators or programmers to control an overall behavior of the network through the control plane while letting the data plane of network devices send and receive packets. Although SDNs pave the way for programming the network, it is difficult to do so because of the complex states and interactions between different network layers. In his paper ``Languages for Software-Defined Networks'', the authors describe the Frenetic project that aims to ease network programming tasks through its higher-level abstractions. From my perspective, computer network researchers can apply formal verification to network protocols and algorithms to make sure that they work correctly and consistently. My research background in computer networks and knowledge of the theory of computation will help me explore and accomplish work in this research area.

\vspace{0.2cm}
Third, Professor Deborah Estrin is working on interesting interdisciplinary research projects. She is bringing back people's digital traces or ``small data'' that they leave on social networks, emails, and mobile applications. According to her talk at TEDMED 2013, these data can be leveraged to promote their health through an open architecture called ``Open mHealth''. For example, digital traces of a patient may include his wakeup time and bedtime, his check-ins at different locations at different time, and the pictures of his food consumed over the weeks before seeing the doctor. All of these can provide his doctor with more useful and penetrable information than the simplistic question ``how have you been?'' and its answer ``I have been fine.'' Research questions and topics of this research area are still open. For example, how can we guarantee an agreeable level of privacy between data owners and service providers while the data still remain meaningful? How can we analyze or mine such a combination of heterogeneous data such as texts, photos, temporal and spacial information? How can we display or visualize these data so that doctors can interpret or understand them more easily? This novel research area which applies mobile data analytics to healthcare will make a tangible and beneficial impact not only on research communities, but also on people in general, including my ageing parents.

\vspace{0.2cm}
My plan after graduation with a doctoral degree is that I will look for a research or post-doc position that is related to the field of my dissertation in order to continue to accumulate research knowledge and experience. Therefore, within five years after graduation, I will become a real expert in the field and plan to lead my own research laboratory. Research experience gained during the PhD study and accumulated after graduation will play an important role in attracting funds and students into my lab.

\vspace{0.2cm}
I would like to express my appreciation to the graduate admission committee of \university \space for taking my statement of purpose along with other application materials into consideration. I hope that the committee will be convinced that my educational background, academic and professional experience, and research ambition and motivation are the evidences sufficient to suggest that I will be an excellent student of the PhD program and a competent researcher in computer science.

\end{document}