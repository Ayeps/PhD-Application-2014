\documentclass[a4paper,10pt]{report}
\hyphenation{mathe-matics under-graduate know-ledge}
\usepackage{fullpage}
\usepackage{fancyhdr}
\pagestyle{fancy}
\fancyhead[L]{\small Kittipat Apicharttrisorn}
\fancyhead[C]{\bfseries  \large Statement of Purpose}
\fancyhead[R]{\tiny \degree \space in Computer Science \\ \university}
\fancyfoot[C]{\thepage}
\topmargin = 10pt
\headheight = 14pt
\headsep = 14pt
\footskip = 30pt
%\usepackage{helvetica}
\newcommand{\university}{Georgia Institute of Technology}
\newcommand{\department}{School of Computer Science}
\newcommand{\uniabbre}{Georgia Tech}
\newcommand{\degree}{Doctor of Philosophy}
\newcommand{\labfirst}{N/A}
\newcommand{\labfirstabbre}{N/A}
\begin{document}
%Briefly describe your eventual career objective. Your space is limited to 150 characters (e.g., University Professor, Industry Researcher, etc.).
\section{career objective}
My plan after graduation with a doctoral degree is that I will look for a research or post-doc position that is related to the field of my dissertation in order to continue to accumulate research knowledge and experience. However, my eventual professional goal is a university professor in the field and plan to lead my own research laboratory. Research experience gained during the PhD study and accumulated during my tenure will play an important role in attracting funds and students into my lab.
%Please describe your background (academic and extracurricular) and experience, including research, teaching, industry, and other relevant information. Your space is limited to 2000 characters.
\section{background}
During my undergraduate study, in addition to a number of Electrical Engineering subjects, I studied a wide range of mathematical subjects including four Calculus courses, a course on Probability, and another on Linear Algebra and Complex Numbers, all of which are basic principles of computer science. Moreover, I passed two courses on computer programming, data structures and algorithms, which are the crucial knowledge of a successful computer scientist. For the Master's degree study, I passed eight credited computer science graduate courses. I studied two theoretical courses, namely Theory of Computation and Computer Algorithms, five systems courses namely Information Systems Architecture, Distributed Systems, Advanced Topics on Computer Networks (Multimedia, Wireless and Adhoc Networks), Embedded Systems, and Database Management Systems, and one Artificial Intelligence course. Moreover, I passed two non-credit courses - namely Computer Security and Special Topics on Distributed Systems (Service Computing). In sum, I earn a solid foundation in computer science as a result of my undergraduate and graduate study. 

\vspace{0.2cm}
In addition, I gain valuable research experience during my Master's degree study and I would like to explain three principal research skills in this letter. First, I learn the critical reading skills. As an important part of research methodology in computer science, literature reviewing is an everyday activity of graduate students. Researchers study research papers not only to understand the overall concepts but also to critique them, find weak points and discover hidden assumptions. With this critical reading, I can find a research opportunity hidden in a research paper and can think of ``what to do next'' instead of just ``this work is interesting''. Second, I learn how to give an intelligible academic presentation. At the UbiNet lab under the supervision of Assistant Professor Dr. Chalermek Intanagonwiwat, each lab member took turns giving one progress presentation reporting the progress toward the thesis work and one paper presentation illustrating the ideas and results of a research paper of interest. Through this regular lab activity, I learned to select an interesting paper published in a well-known conference or journal publisher, to extract outstanding points in the paper and to present them in a way that made it easier for the audience to understand. Third, after completing a certain amount of literature review and implementation work, I need to publish a paper in order to organize my ideas into a standard format, to distribute my work for other researchers to study and to welcome feedbacks and comments from reviewers which help strengthen my work. According to my advisor, a high-quality paper in computer science should not only allow the readers to understand the overall picture of the work, but also enable them to implement it into the code themselves. Therefore, I learn to explain the data structures, algorithms, and communication packets so clearly that one could use all this information for further experimentation. In sum, I earn research experience and skills not through lectures or workshops but by application and repetition throughout the years of the Master's degree study.

\vspace{0.2cm}
Up to now, I have published three academic publications, two of which are in international conferences' proceedings and the other is in an ACM journal.  First, ``Energy-Efficient Gradient Time Synchronization for Wireless Sensor Networks'' was published in the proceedings of the Second International Conference on Computational Intelligence, Communication Systems and Networks or CICSyN. In the paper, we designed an extended version of gradient time synchronization protocols that was more time-accurate and energy-efficient, while maintaining a ``gradient'' property. With the gradient property, geographically adjacent nodes are able to maintain minimal synchronization errors. Second, ``Desynchronization with an artificial force field for wireless networks'' was published in ACM SIGCOMM's \textit{Computer Communication Review}. The desynchronization problem was analogous to a resource allocation problem in which nodes cooperated to take turns accessing to the same resource. In this paper, we provided a prove of convexity of this problem. Additionally, we designed a desynchronization protocol, inspired by electromagnetic force field, that performed in a distributed manner, better scaled with network sizes and densities and produced less desynchronization errors. The first two papers were my work under the supervision of Assistant Professor Dr. Chalermek Intanagonwiwat. Third, in 2013, I had a change to work on a research project with Associate Professor Dr. Teerasit Kasetkasem of Kasetsart University. In this project, we used a signal processing technique to track a moving object in a field given binary sensor observations. In this paper, I was fully responsible for the manuscript preparation and partly for experimental simulation. Finally, the paper titled ``A Moving Object Tracking Algorithm Using Support Vector Machines in Binary Sensor Networks'', was finally published in the proceedings of The 13th International Symposium on Communications and Information Technologies.

\vspace{0.2cm}
I also have seven-year professional experience working at Aeronautical Radio of Thailand or Aerothai, a state enterprise under the Ministry of Transport, Thailand. One of Aerothai's principal missions is to provide safe and efficient air navigation services or air traffic control within Thailand's airspace. Specifically, the department of air traffic data systems engineering is responsible for the provision and administration of data systems that support air traffic controllers' operations. At the department, my colleagues and I design, configure, and implement those systems by taking advantage of enterprise-graded computing system products, mostly of the USA, such as HP and Dell servers, Oracle and Microsoft databases, Cisco network equipment, and VMWare's virtualization technology, etc. One of the interesting aeronautical applications that runs on these infrastructures is the flight scheduling service, named Bay of Bengal Cooperative Air Traffic Flow Management System or BOBCAT. BOBCAT manages the air traffic over the Bay of Bengal, which has the security constraints. Approximately 60 flights per day request to fly through this narrow airspace; therefore, International Civil Aviation Organization or ICAO demands that the airspace be managed by Aerothai, after the systems competition with other organizations. Nowadays, BOBCAT smoothly serves tens of airline customers requesting air space slots over the area every day thanks to Aerothai's effective software systems and responsive operational procedures.  Therefore, I have witnessed how these innovative products help enhance reliability and efficiency of air traffic data systems. This hand-on experience has provided me with practical aspects of enterprise information systems with the safety-critical applications, and motivates me to study more deeply and broadly in computer science, a core foundation of computer-related products and services. 
%Please give a Statement of Purpose detailing your academic and research goals as well as career plans. Include your reasons for choosing the Computational Science and Engineering Program as opposed to other programs and/or other universities. Your space is limited to 4000 characters.

\section{Statement of Purpose }
This statement of purpose is intended for use with my application to the \degree \space graduate program at the \department, \university. In this document, I explain my motivation for the PhD study, my interest in \uniabbre's teaching and research and my future plans after Ph.D. graduation. After finishing reading this statement of purpose, the committee will learn why I am qualified to be an excellent student of the program, what motivates me to pursue the doctoral degree at \uniabbre \space and why it is so important for my future profession that I earn this degree.

\vspace{0.2cm}
I determine to advance my study to a PhD in the US because of the following three main reasons. First and most importantly, I want to be a professional researcher in computer science in the future, either in an academic institution or in a research laboratory and a doctoral degree is an important precursor to the research profession. Second, I agree with Matt Welsh, previously a professor of Computer Science at Harvard University, about a PhD study. He suggests that ``You get an intense exposure to every subfield of Computer Science, and have to become the leading world's expert in the area of your dissertation work.'' For example, during my PhD study, I will have an opportunity to get exposed to a variety of academic subjects and research projects in computer science, such as Artificial Intelligence, Computer Graphics, Robotics, Databases, Systems, Software Engineering, and Computational Science, etc., all of which will considerably expand my intellectual horizons in computer science. Moreover, the PhD study will train me to be an expert in the field of my dissertation through the educational systems and processes, and through my assiduous and persevering efforts. Third, I am conscious that studying at a PhD level requires an academically vibrant environment which includes surroundings with brilliant students and faculty members, as well as accessible academic conferences and seminars. In my opinion, all of these are prevalent in the US educational systems and universities.

% the following paragraphs are specific to a particular program in a university. need to change for each submission to each program
\vspace{0.2cm}
I aspire to become a PhD student at the \department, \university, a prestigious university in the US, because I am particularly interested in its teaching and research. To begin with the teaching, the course ``Computer Networks'' is basically of my interest.  In this course, I will have a chance to sharpen my understanding of computer networks, following a classical computer network textbook ``Computer Networking: A Top-Down Approach''. I studied this textbook both for a graduate course and by myself, and found that its top-down approach allowed me to zoom from the big picture of the Internet's applications to individual protocols of each layer in the protocol stack. Second, ``Software Defined Networking'' is an interesting seminar that introduces software defined networking that allows network researchers or administrators to program the networks. It centralizes control and configuration into the control plane while letting the data plane forward the communication packets. Third, the project-based ``Advanced Internet Application Development'' course requires students to aggregate the knowledge of operating systems, database management, computer networks, and distributed systems so that they can use this body of knowledge to develop potential and high-performance Internet applications. 

\vspace{0.2cm}
The following are \uniabbre's faculty whose research projects interest and excite me. First, Associate Professor Nick Feamster is a notable and active researcher in computer networks. His publications continuously appear on conference and journal articles of ACM SIGCOMM, one of the most dominant computer network research communities. For example, ``Measuring and Mitigating Web Performance Bottlenecks in Broadband Access Networks'' is an interesting Internet measurement paper that tries to solve an everyday problem of web latency from home users' perspective. For example, when I am opening the web page www.gatech.edu and facing an annoying latency, I tend to blame either the slow web server or my poor home Internet connection. However, this paper takes advantage of a measurement tool to find the ``root cause'' of the latency; then, it proposes a simple solution of ``home caching''. Its simple concept is to pre-fetch content and information that cause the latency; as a result, a home user experiences faster web performance at the expense of trivial cache storage. His research style is a paragon of solving a practical problem with a simple solution. 

\vspace{0.2cm}
Second, Professor Dr. Ling Liu's research interests span cloud, mobile, and distributed computing. Tens of her publications appear on top conference and journal articles of the research areas every year. For example, her publication ``Enhanced Monitoring-as-a-Service for Effective Cloud Management'' introduces monitor-as-a-service in the cloud with the features of state monitoring, self-adjustment, and monitoring consolidation. One of its main goals is to reduce the monitoring cost. From my perspective as a systems engineer maintaining and monitoring aeronautical data sytems, this issue is important because while I want my monitoring systems to be as responsive to a problem or anomaly as possible, I need to prevent them from using a lot of my production systems resources, such as bandwidth of the networks, and memory and CPU of the servers. With the supposedly large scale of cloud computing, a monitoring with all these features is preferable and needs extensive attention from cloud computing research.

\vspace{0.2cm}
Third, Professor Dr. Calton Pu's interesting Elba research project tackles a problem of adaptive enterprise computing which answers a question of how to evaluate and tune a large-scaled production enterprise systems ``on-the-fly''. A project paper ``Detecting Transient Bottlenecks in n-Tier Applications through Fine-Grained Analysis'' proposes a fine-grained technique to analyze transient bottlenecks in an n-tier systems. Transient bottlenecks occur in such a short time that existing monitoring tools with a granularity of seconds or minutes fail to capture them. In my opinion, my professional experience with practical enterprise systems with safety-critical applications can contribute to this and other related projects of Professor Dr. Calton Pu. In sum, because \uniabbre's teaching and research match my academic interest and goals, the \department \university \space will be a perfect place for my PhD study until my graduation day.

\vspace{0.2cm}
My plan after graduation with a doctoral degree is that I will look for a research or post-doc position that is related to the field of my dissertation in order to continue to accumulate research knowledge and experience. However, my eventual professional goal is a university professor in the field and plan to lead my own research laboratory. Research experience gained during the PhD study and accumulated during my tenure will play an important role in attracting funds and students into my lab.

\vspace{0.2cm}
I would like to express my appreciation to the graduate admission committee of \university \space for taking my personal statement along with other application materials into consideration. I hope that the committee will be convinced that my educational background, academic and professional experience, and research ambition and motivation are the evidences sufficient to suggest that I will be an excellent student of the PhD program and a competent researcher in computer science.

\end{document}